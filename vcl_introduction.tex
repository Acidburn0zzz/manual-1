% chapter included in vclmanual.tex
\documentclass[vcl_manual.tex]{subfiles}

\begin{document}

\chapter{Introduction}\label{chap:Introduction}
\flushleft
The VCL vector class library is a tool that allows C++ programmers to make their code much faster by handling multiple data in parallel. Modern CPU’s have \textit{Single Instruction Multiple Data} (SIMD) instructions for handling vectors of multiple data elements in parallel. The compiler may be able to use SIMD instructions automatically in simple cases, but a human programmer is often able to do it better by organizing data into vectors that fit the SIMD instructions. The VCL library is a tool that makes it easier for the programmer to write vector code without having to use assembly language or intrinsic functions. Let us explain this with an example:
\vspacesmall

\begin{example}
\label{exampleArrayLoop1}
\end{example} % frame disappears if I put this after end lstlisting
\begin{lstlisting}[frame=single]
// Array loop
float a[8], b[8], c[8];        // declare arrays
...                            // put values into arrays
for (int i = 0; i < 8; i++) {  // loop for 8 elements
    c[i] = a[i] + b[i] * 1.5f; // operations on each element
}
\end{lstlisting}
\vspacesmall

The vector class library allows you to rewrite example \ref{exampleArrayLoop1} using vectors:
\vspacesmall

\begin{example}
\label{exampleArrayLoopVect}
\end{example}
\begin{lstlisting}[frame=single]
// Array loop using vectors
#include "vectorclass.h"       // use vector class library
float a[8], b[8], c[8];        // declare arrays
...                            // put values into arrays
Vec8f avec, bvec, cvec;        // define vectors of 8 floats each
avec.load(a);                  // load array a into vector
bvec.load(b);                  // load array b into vector
cvec = avec + bvec * 1.5f;     // do operations on vectors
cvec.store(c);                 // save result in array c
\end{lstlisting}
\vspacesmall

Example \ref{exampleArrayLoopVect} does the same as example \ref{exampleArrayLoop1}, but more efficiently because it utilizes SIMD instructions that do eight additions and/or eight multiplications in a single instruction. Modern microprocessors have these instructions which may give you a throughput of eight floating point additions and eight multiplications per clock cycle. A good optimizing compiler may actually convert example \ref{exampleArrayLoop1} automatically to use the SIMD instructions, but in more complicated cases you cannot be sure that the compiler is able to vectorize your code in an optimal way.
\vspacesmall

\section{How it works} \label{HowItWorks}
The type \codei{Vec8f} in example \ref{exampleArrayLoopVect} is a class that encapsulates the intrinsic type 
\codei{\_\_m256} which represents a 256-bit vector register holding 8 floating point numbers of 32 bits each. The overloaded operators \codei{+} and \codei{*} represent the SIMD instructions for adding and multiplying vectors. These operators are inlined so that no extra code is generated other than the SIMD instructions. All you have to do to get access to these vector operations is to include "vectorclass.h" in your C++ code and specify the desired instruction set (e.g. SSE2 or AVX512) in your compiler options.
\vspacesmall

The code in example \ref{exampleArrayLoopVect} can be reduced to just 4 machine instructions if the instruction set AVX or higher is enabled. The SSE2 instruction set will give 8 machine instructions because the maximum vector register size is only half as big for instruction sets prior to AVX. The code in example \ref{exampleArrayLoop1} will generate approximately 44 instructions if the compiler does not automatically vectorize the code.
\vspacesmall

\section{Features of VCL} \label{Features}

\begin{itemize}
\item Vectors of 8-, 16-, 32- and 64-bit integers, signed and unsigned
\item Vectors of single and double precision floating point numbers
\item Total vector size 128, 256, or 512 bits
\item Defines almost all common operators
\item Boolean operations and branches on vector elements
\item Many arithmetic functions
\item Standard mathematical functions
\item Permute, blend, gather, scatter, and table look-up functions
\item Fast integer division
\item Can build code for different instruction sets from the same source code
\item CPU dispatching to utilize higher instruction sets when available
\item Uses metaprogramming to find the optimal implementation for the selected instruction set and parameter values of a given operator or function
\item Includes extra add-on packages for special purposes and applications
\end{itemize}
\vspacesmall

\section{Instruction sets supported} \label{InstructionSetsSupported}
Since 1997, each new CPU model has extended the x86 instruction set with more SIMD instructions. The VCL library requires the SSE2 instruction set as a minimum, and supports SSE2, SSE3, SSSE3, SSE4.1, SSE4.2, AVX, AVX2, XOP, FMA3, FMA4, and AVX512F/VL/BW/DQ/ER, as well as the future AVX512VBMI/VBMI2.
\vspacesmall

\section{Platforms supported} \label{PlatformsSupported}
Windows, Linux, and Mac, 32-bit and 64-bit, with Intel, AMD, or VIA x86 or x86-64 instruction set processor.
\vspacesmall

A special version of the vector class library for the Intel Knights Corner coprocessor has been developed at CERN. It is available from
https://bitbucket.org/veclibknc/vclknc.git or 
https://bitbucket.org/edanor/umesimd/
\vspacesmall

\section{Compilers supported} \label{CompilersSupported}
The vector class library could not have been made with any other programming language than C++, because only C++ combines all the necessary features: low-level programming such as bit manipulation and intrinsic functions, high-level programming features such as classes and templates, operator overloading, metaprogramming, compiling to machine code without any intermediate byte code, and highly optimizing compilers with support for the many different instruction sets and platforms.
\vspacesmall

The vector class library works with Microsoft, Intel, Gnu, and Clang C++ compilers. It is recommended to use the newest version of the compiler if the newest instruction sets are used.
\vspacesmall

The vector class library version 1.xx uses standard C++0x, while version 2.xx requires C++17 or later.
\vspacesmall

\section{Intended use} \label{IntendedUse}
This vector class library is intended for experienced C++ programmers. It is useful for improving code performance where speed is critical and where the compiler is unable to vectorize the code automatically in an optimal way. Combining explicit vectorization by the programmer with other kinds of optimization done by the compiler, it has the potential for generating highly efficient code. This can be useful for optimizing library functions and critical innermost loops (hotspots) in CPU-intensive programs. There is no reason to use it in less critical parts of a program.
\vspacesmall

\section{How VCL uses metaprogramming} \label{HowVCLUsesMetaprogramming}
The vector class library uses metaprogramming extensively to resolve as much work as possible at compile time rather than at run time. Especially, it uses metaprogramming to find the optimal instructions and algorithms, depending on constants in the code and the selected instruction set.
\vspacesmall

VCL version 1.xx is written for older versions of the C++ language that does not have very good metaprogramming features, but the VCL makes the best use of the available features such as preprocessing directives and templates. Furthermore, it relies extensively on optimizing compilers for  doing calculations with constant inputs at compile time and for removing not-taken branches.
\vspacesmall

VCL version 2.xx is taking advantage of \codei{constexpr} branches, \codei{constexpr} functions, and other advanced features in 
C++14 and C++17 for explicitly telling the compiler what calculations to do at compile time, and to remove not-taken branches. This makes the code clearer and more efficient. It is recommended to use the latest version of VCL, if possible.
\vspacesmall

The following cases illustrate the use of metaprogramming in VCL:
\begin{itemize}
\item Compiling for different instruction sets. If you are using a bigger vector size than supported by the instruction set, then the VCL code will split the big vector into multiple smaller vectors. If you compile the same code again for a higher instruction set, then you will get a more efficient program with full-size vector registers.

\item Permute, blend, and gather functions. There are many different machine instructions that move data between different vector elements. Some of these instructions can only do very specific data permutations. The VCL uses quite a lot of metaprogramming to find the instruction or sequence of instructions that best fits the specific permutation pattern specified. Often, the higher instruction sets give more efficient results.

\item Integer division. Integer division can be done faster by a combination of multiplication and bit-shifting. The VCL can use metaprogramming to find the optimal division method and calculate the multiplication factor and shift count at compile time if the divisor is a known constant. 
See page \pageref{HowVCLUsesMetaprogramming} for details.

\item Raising to a power. Calculating $x^8$ can be done faster by squaring $x$ three times rather than by a loop that multiplies seven times. The VCL can determine the optimal way of raising floating point vectors to an integer or rational power in the functions \codei{pow\_const} and \codei{pow\_rational}.
\end{itemize}
\vspacesmall

\section{Availability} \label{Availability}
The newest version of the vector class library is available from 
\href{https://github.com/vectorclass}{github.com/vectorclass}
\vspacesmall

A discussion board for the vector class library is currently provided at
\href{https://www.agner.org/optimize/vectorclass/}{www.agner.org/optimize/vectorclass}. 
This may later be moved to a more appropriate place.
\vspacesmall


\section{License} \label{License}
The Vector class library is licensed under the Apache License, version 2.0.
\vspacesmall

You may not use the files except in compliance with this License.
You may obtain a copy of the license at
\href{https://www.apache.org/licenses/LICENSE-2.0}{www.apache.org/licenses/LICENSE-2.0}
\vspacesmall


\chapter{The basics}\label{chap:TheBasics}
\section{How to compile} \label{HowToCompile}

Copy the latest version of the header files (*.h) to the same folder as your C++ source files. The header files from any add-on package should be included too if needed.
\vspacesmall

Include the header file vectorclass.h in your C++ source file.
Several other header files will be included automatically.
\vspacesmall

Set your compiler options to the desired instruction set. The instruction set must be at least SSE2. See table \ref{table:CommandLineOptions} on page \pageref{table:CommandLineOptions} for a list of compiler options. It is recommended to compile for 64-bit mode.
You may compile multiple versions for different instruction sets as explained in chapter \ref{CPUDispatching}.
\vspacesmall

%If you are using the Gnu compiler version 3.x or 4.x then you must set the ABI version to 4 or more, or 0 for a reasonable default. 

The following simple C++ example may help you get started:

\begin{example}
\label{exampleArrayLoop3}
\end{example} 
\begin{lstlisting}[frame=single]
// Simple vector class example C++ file
#include <stdio.h>
#include "vectorclass.h"

int main() {
    // define and initialize integer vectors a and b
    Vec4i a(10,11,12,13);
    Vec4i b(20,21,22,23);

    // add the two vectors
    Vec4i c = a + b;

    // Print the results
    for (int i = 0; i < c.size(); i++) {
        printf(" %5i", c[i]);
    }
    printf("\n");

    return 0;
}
\end{lstlisting}
\vspacesmall

\section{Overview of vector classes} \label{OverviewOfVectorClasses}
The vector class library supports vectors of 8-bit, 16-bit, 32-bit and 64-bit signed and unsigned integers, 32-bit single precision floating point numbers, and 64-bit double precision floating point numbers. A vector contains multiple elements of the same type to a total size of 128, 256 or 512 bits. The vector elements are indexed, starting at 0 for the first element.
\vspacesmall

The constant MAX\_VECTOR\_SIZE indicates the maximum vector size. The default maximum vector size is 512 in the current version and possibly larger in future versions. You can disable 512-bit vectors by defining
\begin{lstlisting}[frame=none]
    #define MAX_VECTOR_SIZE 256
\end{lstlisting}
before including the vector class header files.

\vspacesmall
The vector class library also defines boolean vectors. These are mainly used for conditionally selecting elements from vectors.

\vspacesmall
The following vector classes are defined:

\begin {table}[H]
\caption{Integer vector classes}
\label{table:integerVectorClasses}
\begin{tabular}{|p{18mm}|p{18mm}|p{18mm}|p{18mm}|p{18mm}|p{30mm}|}
\hline
\bfseries Vector class & \bfseries Integer size bits & \bfseries Signed & \bfseries Elements per vector & \bfseries Total bits & \bfseries Minimum
\newline recommended \newline instruction set \\ \hline
Vec16c  & \centering  8 & signed   & \centering 16 & \centering 128 & SSE2 \\ \hline
Vec16uc & \centering  8 & unsigned & \centering 16 & \centering 128 & SSE2 \\ \hline
Vec8s   & \centering 16 & signed   & \centering  8 & \centering 128 & SSE2 \\ \hline
Vec8us  & \centering 16 & unsigned & \centering  8 & \centering 128 & SSE2 \\ \hline
Vec4i   & \centering 32 &   signed & \centering  4 & \centering 128 & SSE2 \\ \hline
Vec4ui  & \centering 32 & unsigned & \centering  4 & \centering 128 & SSE2 \\ \hline
Vec2q   & \centering 64 &   signed & \centering  2 & \centering 128 & SSE2 \\ \hline
Vec2uq  & \centering 64 & unsigned & \centering  2 & \centering 128 & SSE2 \\ \hline
Vec32c  & \centering  8 &   signed & \centering 32 & \centering 256 & AVX2 \\ \hline
Vec32uc & \centering  8 & unsigned & \centering 32 & \centering 256 & AVX2 \\ \hline
Vec16s  & \centering 16 &   signed & \centering 16 & \centering 256 & AVX2 \\ \hline
Vec16us & \centering 16 & unsigned & \centering 16 & \centering 256 & AVX2 \\ \hline
Vec8i   & \centering 32 &   signed & \centering  8 & \centering 256 & AVX2 \\ \hline
Vec8ui  & \centering 32 & unsigned & \centering  8 & \centering 256 & AVX2 \\ \hline
Vec4q   & \centering 64 &   signed & \centering  4 & \centering 256 & AVX2 \\ \hline
Vec4uq  & \centering 64 & unsigned & \centering  4 & \centering 256 & AVX2 \\ \hline
Vec64c  & \centering  8 &   signed & \centering 64 & \centering 512 & AVX512BW \\ \hline
Vec64uc & \centering  8 & unsigned & \centering 64 & \centering 512 & AVX512BW \\ \hline
Vec32s  & \centering 16 &   signed & \centering 32 & \centering 512 & AVX512BW \\ \hline
Vec32us & \centering 16 & unsigned & \centering 32 & \centering 512 & AVX512BW \\ \hline
Vec16i  & \centering 32 &   signed & \centering 16 & \centering 512 & AVX512 \\ \hline
Vec16ui & \centering 32 & unsigned & \centering 16 & \centering 512 & AVX512 \\ \hline
Vec8q   & \centering 64 &   signed & \centering  8 & \centering 512 & AVX512 \\ \hline
Vec8uq  & \centering 64 & unsigned & \centering  8 & \centering 512 & AVX512 \\ \hline
\end{tabular}
\end{table}
\vspacesmall

\begin {table}[H]
\caption{Floating point vector classes}
\label{table:FloatVectorClasses}
\begin{tabular}{|p{18mm}|p{18mm}|p{18mm}|p{18mm}|p{30mm}|}
\hline
\bfseries Vector class & \bfseries Precision &  \bfseries Elements per vector & \bfseries Total bits & \bfseries  Minimum
\newline recommended \newline instruction set \\ \hline
Vec4f  & \centering single & \centering  4 & \centering 128 & SSE2 \\ \hline
Vec2d  & \centering double & \centering  2 & \centering 128 & SSE2 \\ \hline
Vec8f  & \centering single & \centering  8 & \centering 256 & AVX \\ \hline
Vec4d  & \centering double & \centering  4 & \centering 256 & AVX \\ \hline
Vec16f & \centering single & \centering 16 & \centering 512 & AVX512 \\ \hline
Vec8d  & \centering double & \centering  8 & \centering 512 & AVX512 \\ \hline
\end{tabular}
\end{table}
\vspacesmall


\begin {table}[H]
\caption{Boolean vector classes}
\label{table:BooleanVectorClasses}
\begin{tabular}{|p{18mm}|p{30mm}|p{18mm}|p{18mm}|p{30mm}|}
\hline
\bfseries Boolean vector class & \bfseries For use with &  \bfseries Elements per vector & \bfseries Total size, bits & \bfseries  Minimum
\newline recommended \newline instruction set\\ \hline
Vec16cb  & \centering Vec16c, Vec16uc & \centering  16 & \centering 16 or 128 & SSE2 \\ \hline
Vec8sb   & \centering Vec8s, Vec8us   & \centering   8 & \centering  8 or 128 & SSE2 \\ \hline
Vec4ib   & \centering Vec4i, Vec4ui   & \centering   4 & \centering  8 or 128 & SSE2 \\ \hline
Vec2qb   & \centering Vec2q, Vec2uq   & \centering   2 & \centering  8 or 128 & SSE2 \\ \hline
Vec32cb  & \centering Vec32c, Vec32uc & \centering  32 & \centering 32 or 256 & AVX2 \\ \hline
Vec16sb  & \centering Vec16s, Vec16us & \centering  16 & \centering 16 or 256 & AVX2 \\ \hline
Vec8ib   & \centering Vec8i, Vec8ui   & \centering   8 & \centering  8 or 256 & AVX2 \\ \hline
Vec4qb   & \centering Vec4q, Vec4uq   & \centering   4 & \centering  8 or 256 & AVX2 \\ \hline
Vec64cb  & \centering Vec64c, Vec64uc & \centering  64 & \centering 64 or 512 & AVX512BW \\ \hline
Vec32sb  & \centering Vec32s, Vec32us & \centering  32 & \centering 32 or 512 & AVX512BW \\ \hline
Vec16ib  & \centering Vec16i, Vec16ui & \centering  16 & \centering 16 or 512 & AVX512 \\ \hline
Vec8qb   & \centering Vec8q,  Vec8uq  & \centering   8 & \centering  8 or 512 & AVX512 \\ \hline
Vec4fb   & \centering Vec4f           & \centering   4 & \centering  8 or 128 & SSE2 \\ \hline
Vec2db   & \centering Vec2d           & \centering   2 & \centering  8 or 128 & SSE2 \\ \hline
Vec8fb   & \centering Vec8f           & \centering   8 & \centering  8 or 256 & SSE2 \\ \hline
Vec4db   & \centering Vec4d           & \centering   4 & \centering  8 or 256 & SSE2 \\ \hline
Vec16fb  & \centering Vec16f          & \centering  16 & \centering  16       & AVX512 \\ \hline
Vec8db   & \centering Vec8d           & \centering   8 & \centering   8       & AVX512 \\ \hline
\end{tabular}
\end{table}

The size of the boolean vectors depends on the instruction set (see page \pageref{tableBooleanVectorSizes}).
\vspacebig


\section{Constructing vectors and loading data into vectors} \label{ConstructingVectors}

There are many ways to create vectors and put data into vectors. These methods are listed here.
\vspacesmall

\begin{tabular}{|p{25mm}|p{100mm}|}
\hline
\bfseries Method & default constructor \\ \hline
\bfseries Defined for & all vector classes \\ \hline
\bfseries Description & the vector is created but not initialized. The value is unpredictable \\ \hline
\bfseries Efficiency & good \\ \hline
\end{tabular}
\vspacesmall

\begin{lstlisting}[frame=none]
// Example:
Vec4i a;    // creates a vector of 4 signed integers
\end{lstlisting}
\vspacesmall

\begin{tabular}[l]{|p{25mm}|p{100mm}|}
\hline
\bfseries Method & constructor with one parameter \\ \hline
\bfseries Defined for & all vector classes \\ \hline
\bfseries Description & all elements get the same value \\ \hline
\bfseries Efficiency & good \\ \hline
\end{tabular}
\begin{lstlisting}[frame=none]
// Example:
Vec4i a(5);   // all four elements = 5
\end{lstlisting}

\vspacesmall
\begin{tabular}{|p{25mm}|p{100mm}|}
\hline
\bfseries Method & assignment to scalar \\ \hline
\bfseries Defined for & all vector classes \\ \hline
\bfseries Description & all elements get the same value \\ \hline
\bfseries Efficiency & good \\ \hline
\end{tabular}
\begin{lstlisting}[frame=none]
// Example:
Vec4i a = 6;  // all four elements = 6
\end{lstlisting}

\vspacesmall
\begin{tabular}{|p{25mm}|p{100mm}|}
\hline
\bfseries Method & constructor with one parameter for each vector element \\ \hline
\bfseries Defined for & all integer and floating point vector classes \\ \hline
\bfseries Description & each element gets a specified value. The parameter for element number 0 comes first. \\ \hline
\bfseries Efficiency & good for constant. Medium for variables as parameters \\ \hline
\end{tabular}
\begin{lstlisting}[frame=none]
// Examples:
Vec4i a(10,11,12,13);         // a = (10,11,12,13)
Vec4i b = Vec4i(20,21,22,23); // b = (20,21,22,23)
\end{lstlisting}

\vspacesmall
\begin{tabular}{|p{25mm}|p{100mm}|}
\hline
\bfseries Method & constructor with one parameter for each half vector \\ \hline
\bfseries Defined for & all vector classes if a similar class of half the size exists \\ \hline
\bfseries Description & Concatenates two 128-bit vectors into one 256-bit vector. \newline
Concatenates two 256-bit vectors into one 512-bit vector \\ \hline
\bfseries Efficiency & good \\ \hline
\end{tabular}
\begin{lstlisting}[frame=none]
// Example:
Vec4i a(10,11,12,13);
Vec4i b(20,21,22,23);
Vec8i c(a, b);    // c = (10,11,12,13,20,21,22,23)
\end{lstlisting}

\vspacesmall
\begin{tabular}{|p{25mm}|p{100mm}|}
\hline
\bfseries Method & insert(index, value) \\ \hline
\bfseries Defined for & all vector classes \\ \hline
\bfseries Description & changes the value of element number (index) to (value). The index starts at 0 \\ \hline
\bfseries Efficiency & good if AVX512VL, medium otherwise \\ \hline
\end{tabular}
\begin{lstlisting}[frame=none]
// Example:
Vec4i a(0);
a.insert(2, 9);  // a = (0,0,9,0)
\end{lstlisting}

\vspacesmall
\begin{tabular}{|p{25mm}|p{100mm}|}
\hline
\bfseries Method & load(const pointer) \\ \hline
\bfseries Defined for & all integer and floating point vector classes \\ \hline
\bfseries Description & loads all elements from an array \\ \hline
\bfseries Efficiency & good, except immediately after inserting elements one by one into the array \\ \hline
\end{tabular}
\begin{lstlisting}[frame=none]
// Example:
int list[8] = {10,11,12,13,14,15,16,17};
Vec4i a, b;
a.load(list);   // a = (10,11,12,13)
b.load(list+4); // b = (14,15,16,17)
\end{lstlisting}

\vspacesmall
\begin{tabular}{|p{25mm}|p{100mm}|}
\hline
\bfseries Method & load\_a(const pointer) \\ \hline
\bfseries Defined for & all integer and floating point vector classes \\ \hline
\bfseries Description & loads all elements from an aligned array \\ \hline
\bfseries Efficiency & good, except immediately after inserting elements separately into the array. \\ \hline
\end{tabular}

This method does the same as the \codei{load} method (see above), but requires that the pointer points to an address divisible by 16 for 128-bit vectors, by 32 for 256-bit vectors, or by 64 for 512 bit vectors. If you are not certain that the array is properly aligned then use \codei{load} instead of \codei{load\_a}. There is hardly any difference in efficiency between \codei{load} and \codei{load\_a} on newer microprocessors.
\vspacebig

\begin{tabular}{|p{25mm}|p{100mm}|}
\hline
\bfseries Method & load\_partial(int n, const pointer) \\ \hline
\bfseries Defined for & all integer and floating point vector classes \\ \hline
\bfseries Description & loads n elements from an array into a vector. Sets remaining elements to 0. 0 $\leq$ n $\leq$ (vector size). \\ \hline
\bfseries Efficiency & good if AVX512VL, medium otherwise \\ \hline
\end{tabular}
\begin{lstlisting}[frame=none]
// Example:
float list[3] = {1.0f, 1.1f, 1.2f};
Vec4f a;
a.load_partial(2, list);  // a = (1.0, 1.1, 0.0, 0.0)
\end{lstlisting}


\vspacesmall
\begin{tabular}{|p{25mm}|p{100mm}|}
\hline
\bfseries Method & cutoff(int n) \\ \hline
\bfseries Defined for & all integer and floating point vector classes \\ \hline
\bfseries Description & leaves the first n elements unchanged and sets the remaining elements to zero. 0 $\leq$ n $\leq$ (vector size). \\ \hline
\bfseries Efficiency & good \\ \hline
\end{tabular}
\begin{lstlisting}[frame=none]
// Example:
Vec4i a(10, 11, 12, 13);
a.cutoff(2);              // a = (10, 11, 0, 0)
\end{lstlisting}


\vspacesmall
\begin{tabular}{|p{25mm}|p{100mm}|}
\hline
\bfseries Method & gather\textless indexes\textgreater (array) \\ \hline
\bfseries Defined for & floating point vector classes and integer vector classes with 32-bit and 64-bit elements \\ \hline
\bfseries Description & gather non-contiguous data from an array. \\ \hline
\bfseries Efficiency & medium \\ \hline
\end{tabular}
\begin{lstlisting}[frame=none]
// Example:
int list[8] = {10,11,12,13,14,15,16,17};
Vec4i a = gather4i<0,2,1,6>(list); // a = (10,12,11,16)
\end{lstlisting}
\vspacesmall

\section{Getting data from vectors} \label{GettingDataFromVectors}

There are many ways to extract elements or parts of a vector. These methods are listed here.

\vspacesmall
\begin{tabular}{|p{25mm}|p{100mm}|}
\hline
\bfseries Method & store(pointer) \\ \hline
\bfseries Defined for & all integer and floating point vector classes \\ \hline
\bfseries Description & stores all elements into an array \\ \hline
\bfseries Efficiency & good \\ \hline
\end{tabular}
\begin{lstlisting}[frame=none]
// Example:
Vec4i a(10,11,12,13);
Vec4i b(20,21,22,23);
int list[8];
a.store(list);
b.store(list+4); // list contains (10,11,12,13,20,21,22,23)
\end{lstlisting}


\vspacesmall
\begin{tabular}{|p{25mm}|p{100mm}|}
\hline
\bfseries Method & store\_a(pointer) \\ \hline
\bfseries Defined for & all integer and floating point vector classes \\ \hline
\bfseries Description & stores all elements into an aligned array \\ \hline
\bfseries Efficiency & good \\ \hline
\end{tabular}
\vspacesmall

This method does the same as the store method (see above), but requires that the pointer points to an address divisible by 16 for 128-bit vectors, by 32 for 256-bit vectors, or by 64 for 512-bit vectors. If you are not certain that the array is properly aligned then use \codei{store} instead of \codei{store\_a}.
There is hardly any difference in efficiency between \codei{store} and \codei{store\_a} on newer microprocessors.


\vspacebig
\begin{tabular}{|p{25mm}|p{100mm}|}
\hline
\bfseries Method & store\_partial(int n, pointer) \\ \hline
\bfseries Defined for & all integer and floating point vector classes \\ \hline
\bfseries Description & stores the first n elements into an array. The rest of the array is untouched. 
0 $\leq$ n $\leq$ (vector size) \\ \hline
\bfseries Efficiency & good if AVX512VL, medium otherwise \\ \hline
\end{tabular}
\begin{lstlisting}[frame=none]
// Example:
float list[4] = {9.0f, 9.0f, 9.0f, 9.0f};
Vec4f a(1.0f, 1.1f, 1.2f, 1.3f);
a.store_partial(2, list);  // list contains (1.0, 1.1, 9.0, 9.0)
\end{lstlisting}

\vspacesmall
\begin{tabular}{|p{25mm}|p{100mm}|}
\hline
\bfseries Method & extract(index) \\ \hline
\bfseries Defined for & all integer, floating point and boolean vector classes \\ \hline
\bfseries Description & gets a single element from a vector \\ \hline
\bfseries Efficiency & good if AVX512VL, medium otherwise \\ \hline
\end{tabular}
\begin{lstlisting}[frame=none]
// Example:
Vec4i a(10,11,12,13);
int b = a.extract(2);   // b = 12
\end{lstlisting}

\vspacesmall
\begin{tabular}{|p{25mm}|p{100mm}|}
\hline
\bfseries Method & operator [ ] \\ \hline
\bfseries Defined for & all integer, floating point and boolean vector classes \\ \hline
\bfseries Description & gets a single element from a vector \\ \hline
\bfseries Efficiency & good if AVX512VL, medium otherwise \\ \hline
\end{tabular}

The operator [ ] does exactly the same as the extract method. Note that you can read a vector element with the [ ] operator, but not write an element.
\vspacesmall

\begin{lstlisting}[frame=none]
// Example:
Vec4i a(10,11,12,13);
int b = a[2];           // b = 12
a[3] = 5;               // not allowed!
\end{lstlisting}

\vspacesmall
\begin{tabular}{|p{25mm}|p{100mm}|}
\hline
\bfseries Method & get\_low() \\ \hline
\bfseries Defined for & all vector classes of 256 bits or more \\ \hline
\bfseries Description & gets the lower half of a 256-bit vector as a 128-bit vector.\newline
gets the lower half of a 512-bit vector as a 256-bit vector.
 \\ \hline
\bfseries Efficiency & good \\ \hline
\end{tabular}
\begin{lstlisting}[frame=none]
// Example:
Vec8i a(10,11,12,13,14,15,16,17);
Vec4i b = a.get_low();  // b = (10,11,12,13)
\end{lstlisting}

\vspacesmall
\begin{tabular}{|p{25mm}|p{100mm}|}
\hline
\bfseries Method & get\_high() \\ \hline
\bfseries Defined for & all vector classes of 256 bits or more \\ \hline
\bfseries Description & gets the upper half of a 256-bit vector as a 128-bit vector.\newline
gets the upper half of a 512-bit vector as a 256-bit vector.
 \\ \hline
\bfseries Efficiency & good \\ \hline
\end{tabular}
\begin{lstlisting}[frame=none]
// Example:
Vec8i a(10,11,12,13,14,15,16,17);
Vec4i b = a.get_high();  // b = (14,15,16,17)
\end{lstlisting}

\vspacesmall
\begin{tabular}{|p{25mm}|p{100mm}|}
\hline
\bfseries Method & size() \\ \hline
\bfseries Defined for & all vector classes \\ \hline
\bfseries Description & static constant member function indicating the number of elements that the vector can contain \\ \hline
\bfseries Efficiency & good \\ \hline
\end{tabular}
\begin{lstlisting}[frame=none]
// Example:
Vec8f a;
int s = a.size();  // s = 8
\end{lstlisting}

\vspacesmall
\begin{tabular}{|p{25mm}|p{100mm}|}
\hline
\bfseries Method & elementtype() \\ \hline
\bfseries Defined for & all vector classes \\ \hline
\bfseries Description & static constant member function indicating the type of elements that the vector contains: \newline
1: bits (internal base class) \newline
2: bool (compact) \newline
3: bool (broad) \newline
4: int8\_t \newline
5: uint8\_t \newline
6: int16\_t \newline
7: uint16\_t \newline
8: int32\_t \newline
9: uint32\_t \newline
10: int64\_t \newline
11: uint64\_t \newline
16: float \newline
17: double \\ \hline
\bfseries Efficiency & good \\ \hline
\end{tabular}

\begin{lstlisting}[frame=none]
// Example:
Vec16s a;
int t = a.elementtype();  // t = 6
\end{lstlisting}


%\indenton   % undo \flushleft
\vspacesmall


\section{Arrays of vectors} \label{ArraysOfVectors}

If you make an array of vectors, this should preferably have fixed size. For example:
\begin{lstlisting}[frame=none]
const int datasize = 1024;  // size of data set
Vec8f mydata[datasize/8];   // array of fixed size
...
for (int i = 0; i < datasize/8; i++) {
   mydata[i] = mydata[i] * 0.1f + 2.0f;
}
\end{lstlisting}
\vspacesmall

If you need an array of a size that is determined at runtime, then you will have a problem with alignment. Each vector should preferably be stored at an address divisible by 16, 32 or 64 bytes, according to its size. The compiler can do this when defining a fixed-size array, as in the above example, but not necessarily with dynamic memory allocation. If you create an array of dynamic size by using new, malloc or a container class template, or any other method, then you may not get the proper alignment for vectors.
A misaligned vector will likely cause the program to crash if compiled for an instruction set less than AVX. See page \pageref{Alignment} for details.
\vspacesmall

It is recommended to make an array of scalars instead of an array of vectors in order to avoid these complications. If the datasize in the above example is variable, then the code could be implemented in this way:

\begin{lstlisting}[frame=none]
int datasize = 1024;  // size of dataset, variable
float *mydata = new float[datasize]; // dynamic array
...
Vec8f x;
for (int i = 0; i < datasize; i += 8) {
   x.load(mydata+i);
   x = x * 0.1f + 2.0f;
   x.store(mydata+i);
}
...
delete[] mydata;
\end{lstlisting}
\vspacesmall

See page \pageref{NotAMultipleOfVectorSize} for discussion of the case where the data size is not a multiple of the vector size.
\vspacesmall

\section{Using a namespace} \label{UsingANamespace}

In general, there is no need to put the vector class library into a separate namespace. Therefore, the use of a namespace is optional. You can give the vector class library a namespace with a name of your choosing by defining \codei{VCL\_NAMESPACE}, for example:

\begin{lstlisting}[frame=single]
#define VCL_NAMESPACE  vcl
#include "vectorclass.h"

using namespace vcl;

// your vector code here...
\end{lstlisting}
\vspacesmall

\end{document}